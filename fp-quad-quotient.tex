\documentclass{book}
\usepackage{amsthm,amsmath,amsfonts,amssymb,tcolorbox,enumitem,hyperref}

\newcommand{\I}[1]{\mathcal{I}(#1)}
\newcommand{\V}[1]{\mathcal{V}(#1)}
\newcommand{\kn}{k[x_1,x_2,\ldots,x_n]}
\newcommand{\Rn}{R[x_1,x_2,\ldots,x_n]}
\newcommand{\Af}{{\Bbb A}^n}
\newcommand{\Ak}{{\Bbb A}_k^n}
\newcommand{\Akn}[1]{{\Bbb A}_k^#1}
\newcommand{\fr}[1]{\mathfrak #1}
\newcommand{\C}{\mathbb C}
\newcommand{\F}{\mathbb F}
\DeclareMathOperator{\im}{im}

\theoremstyle{definition}
\newtheorem*{lem}{Lemma}
\newtheorem*{cor}{Corollary}
\newtheorem*{thmsmall}{Theorem}

\newtheoremstyle{block}
 {\topsep}% space above
 {\topsep}% space below
 {}% body font
 {}% indent
 {\bfseries}% head font
 {}% punctuation
 {1em}% space between head and body
 {Problem.}

\theoremstyle{block}
\newtheorem{block}{}
\newtheorem{block*}[block]{}

\newtheoremstyle{thm}
 {\topsep}% space above
 {\topsep}% space below
 {}% body font
 {0em}% indent
 {\bfseries}% head font
 {}% punctuation
 {0em}% space between head and body
 {#3. }

\theoremstyle{thm}
\newtheorem*{thm}{}

\tcolorboxenvironment{block*}{
  frame empty,
  colback=black!15!white,
  grow to left by=6pt,
  grow to right by=6pt,
  left=1.6pt,
  right=1.6pt,
  arc=0pt
}

\tcolorboxenvironment{thm}{
  frame empty,
  colback=black!15!white,
  grow to left by=6pt,
  grow to right by=6pt,
  left=1.6pt,
  right=1.6pt,
  arc=0pt
}

\begin{document}

\begin{block*}
  Show that there are exactly three isomorphism classes of quotient rings of
  $\F_p[x]$ by (ideals generated by) quadratic polynomials.
\end{block*}

\begin{proof}
  Let the polynomial be $f = x^2+mx+n$. We divide into cases based on whether or not
  $f$ is reducible in $\F_p[x]$:
  \begin{enumerate}
    \item $f$ is irreducible. Then $(f)$ is maximal, and hence the quotient ring
      $\F_p[x]/(f)$ is a field, of order $p^2$ (since, by the division
      algorithm, every ``coset'' will correspond to a linear polynomial, and there
      are $p^2$ of those). Since there is a unique finite field for every prime
      power order, $\F_{p^2}$ forms one isomorphism class.
    \item $f$ is reducible and factors as $f = (x+a)(x+b)$. We divide further:
      \begin{enumerate}
        \item $a \neq b$. Then
          \[\frac{\F_p[x]}{\langle {(x+a)(x+b)} \rangle} \cong
            \frac{\F_p[x]}{(x+a)}\times\frac{\F_p[x]}{(x+b)} \cong
            \F_p\times\F_p\]
          which gives us another isomorphism class of quotient rings.
          The first isomorphism holds because both factors are prime, and the
          second isomorphism is obtained by applying the first isomorphism
          theorem to the map $\text{ev}_{\gamma}:\F_p[x]\to\F_p$, which is surjective
          and has kernel $(x-\gamma)$, for $\gamma = -a$.
        \item $a = b$. Then
          \[\frac{\F_p[x]}{\langle (x+a)^2 \rangle}\cong \F_p[x]/(x^2)
          \]
          with an explicit isomorphism given by $\overline{f(x)}\mapsto
          \overline{f(x-a)}$. 
      \end{enumerate}
  \end{enumerate}
\end{proof}

\end{document}