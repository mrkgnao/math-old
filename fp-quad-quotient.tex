\documentclass{book}
\usepackage{amsthm,amsmath,amsfonts,amssymb,tcolorbox,enumitem,hyperref}

\newcommand{\I}[1]{\mathcal{I}(#1)}
\newcommand{\V}[1]{\mathcal{V}(#1)}
\newcommand{\kn}{k[x_1,x_2,\ldots,x_n]}
\newcommand{\Rn}{R[x_1,x_2,\ldots,x_n]}
\newcommand{\Af}{{\Bbb A}^n}
\newcommand{\Ak}{{\Bbb A}_k^n}
\newcommand{\Akn}[1]{{\Bbb A}_k^#1}
\newcommand{\fr}[1]{\mathfrak #1}
\newcommand{\C}{\mathbb C}
\DeclareMathOperator{\im}{im}

\theoremstyle{definition}
\newtheorem*{lem}{Lemma}
\newtheorem*{cor}{Corollary}
\newtheorem*{thmsmall}{Theorem}

\newtheoremstyle{block}
 {\topsep}% space above
 {\topsep}% space below
 {}% body font
 {}% indent
 {\bfseries}% head font
 {}% punctuation
 {1em}% space between head and body
 {(\thmnumber{#2})}

\theoremstyle{block}
\newtheorem{block}{}[section]
\newtheorem{block*}[block]{}

\newtheoremstyle{thm}
 {\topsep}% space above
 {\topsep}% space below
 {}% body font
 {0em}% indent
 {\bfseries}% head font
 {}% punctuation
 {0em}% space between head and body
 {#3. }

\theoremstyle{thm}
\newtheorem*{thm}{}

\tcolorboxenvironment{block*}{
  frame empty,
  colback=black!15!white,
  grow to left by=6pt,
  grow to right by=6pt,
  left=1.6pt,
  right=1.6pt,
  arc=0pt
}

\tcolorboxenvironment{thm}{
  frame empty,
  colback=black!15!white,
  grow to left by=6pt,
  grow to right by=6pt,
  left=1.6pt,
  right=1.6pt,
  arc=0pt
}

\begin{document}

\begin{block*}
  Show that the localization of a ring $R$ at a prime ideal $\fr p$ is a local
  ring. What is the maximal ideal?
\end{block*}
\begin{proof}[Solution]
  I claim that the unique maximal ideal is the ideal of nonunits of $R_{\fr p}$,
  more usually defined as
  \[
    \fr m = \{\frac a b : \text{some element of } \fr p \text { divides } a\}
  \]
  For any other ideal $\fr i$, there are two options:
  \begin{enumerate}[label=(\alph*)]
    \item It contains only nonunits. Then obviously $\fr i \subset \fr m$.
    \item It contains some element that is a unit. Since ideals are closed under
      multiplication, multiply by the inverse of the unit to get a $1$ in the
      ideal, implying that $\fr i = (1)$.
  \end{enumerate}
  Hence all proper ideals are contained in $\fr m$.
\end{proof}

\chapter{Rings and ideals}

\section{Nilradical and Jacobson radical}
Define the \textit{nilradical} of a ring $R$ as \[N_R:=\sqrt{(0)_R}\]
equivalently, $N(R)$ is the ``ideal of nilpotents''. The main use of the
nilradical of a ring (as far as I know!) is to eliminate the problems associated
with nilpotents by looking at $R/N_R$ instead of $R$: it seems plausible (at
least intuitively) that this ring has none.

\begin{block}
  $N_R$ is an ideal.
\end{block}
\begin{proof}
  Closure under $R$-multiplication is clear. As for additive closure, consider
  $x,y\in N_R$, such that $x^m = y^n = 0$. We must show that $x+y$ is also a
  nilpotent.\par
  Consider $(x+y)^{m+n-1}$. Every term of the binomial expansion contains either
  a power of $x^{r\geq m}$ or $y^{s\geq n}$ (else $r+s<m+n-1$), so the entire thing is zero.
\end{proof}

\begin{block}
  $R/N_R$ has no nonzero nilpotents.
\end{block}
\begin{proof}
  Suppose $\bar x$ is a nilpotent in $R/N_R$, with ${\bar x}^n=0$. Then $x^n$ is
  a nilpotent in $R$ and is mapped to $(0)\in R/N_R$, implying that $\bar x = 0$. 
\end{proof}

\begin{block}
  The nilradical of a ring $R$ is the intersection of all its prime ideals.
\end{block}
\begin{proof}
  Let $\fr P$ denote the intersection of all the prime ideals. If $x\in R$ is
  nilpotent and $\fr p$ is a prime ideal, then (for some $n$) $x^n = 0\in\fr p$,
  hence $x\in\sqrt{\fr p}=\fr p$. Hence $N_R\subset\fr p$.\par
  Now we prove that $x$ not nilpotent $\implies x\notin\fr P$. Let $\Sigma$ be
  the set of ideals whose radicals do not contain $x$. Since $(0)\in\Sigma$, it
  is nonempty and can be ordered by inclusion. Applying Zorn's lemma, we get a
  maximal element $\fr p$ of $\Sigma$. We will show that $\fr p$ is prime.\par
  Choose $s,t\notin\fr p$. Then $\fr p + (s), \fr p + (t)$ strictly contain $\fr
  p$, so they are not in $\Sigma$. By the definition of $\Sigma$, this means
  that some power of $x$ is in both those sum ideals:
  \[x^m\in\fr p + (s), x^n\in\fr p + (t)\]
  for some $m$ and $n$. Then
  \[x^{m+n}\in \fr p + (xy)\]
  so $\fr p + (xy)$ is not in $\Sigma$ either, so $xy\notin\fr p$. Hence we have
  a prime ideal $\fr p$ not containing $x$, so $x\notin\fr P$ either.
\end{proof}

\begin{proof}[\href{http://chat.stackexchange.com/transcript/message/24368544\#24368544}{Balarka's
  proof}] 
  Assume $x$ in $A$ is not nilpotent. Then $A_x$, the localization at powers of
  $x$, is nonzero. Pick a maximal ideal $\mathfrak{m}$ (here lies disguised
  Zorn). Contract to get a prime ideal in $A$ not containing any power of $x$.
  Hence, $x$ is not contained in a prime ideal of $A$ and thus not in intersection
  of all prime ideals in $A$.
\end{proof}
\end{document}