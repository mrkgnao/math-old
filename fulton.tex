\documentclass{book}
\usepackage{amsthm,amsmath,amsfonts,amssymb,tcolorbox,enumitem}

\newcommand{\I}[1]{\mathcal{I}(#1)}
\newcommand{\V}[1]{\mathcal{V}(#1)}
\newcommand{\kn}{k[x_1,x_2,\ldots,x_n]}
\newcommand{\Rn}{R[x_1,x_2,\ldots,x_n]}
\newcommand{\Af}{{\Bbb A}^n}
\newcommand{\Ak}{{\Bbb A}_k^n}
\newcommand{\Akn}[1]{{\Bbb A}_k^#1}
\newcommand{\fr}[1]{\mathfrak #1}
\newcommand{\C}{\mathbb C}
\DeclareMathOperator{\im}{im}

\theoremstyle{definition}
\newtheorem*{lem}{Lemma}
\newtheorem*{cor}{Corollary}
\newtheorem*{thmsmall}{Theorem}

\newtheoremstyle{block}
 {\topsep}% space above
 {\topsep}% space below
 {}% body font
 {}% indent
 {\bfseries}% head font
 {}% punctuation
 {1em}% space between head and body
 {(\thmnumber{#2})}

\theoremstyle{block}
\newtheorem{block}{}[section]
\newtheorem{block*}[block]{}

\newtheoremstyle{thm}
 {\topsep}% space above
 {\topsep}% space below
 {}% body font
 {0em}% indent
 {\bfseries}% head font
 {}% punctuation
 {0em}% space between head and body
 {#3. }

\theoremstyle{thm}
\newtheorem*{thm}{}

\tcolorboxenvironment{block*}{
  frame empty,
  colback=black!15!white,
  grow to left by=6pt,
  grow to right by=6pt,
  left=1.6pt,
  right=1.6pt,
  arc=0pt
}

\tcolorboxenvironment{thm}{
  frame empty,
  colback=black!15!white,
  grow to left by=6pt,
  grow to right by=6pt,
  left=1.6pt,
  right=1.6pt,
  arc=0pt
}

\begin{document}

\chapter{Affine algebraic sets}

\section{The ideal of a set of points}
\begin{block*}
  Prove the following properties:
  \begin{enumerate}[label=(\alph*)]
    \item $\mathcal I$ and $\mathcal V$ reverse inclusions. 
    \item $\I{\V S}\supset S$ for all $S\subset\kn$. Dually, $\V{\I X}\supset X$
      for all $V\subset\Af$.
    \item $\V{\I{\V S}}=\V S$ for $S$ a set of polynomials; $\I{\V{\I X}}=\I X$
      for $X$ a set of points.
\end{enumerate}
\end{block*}

\begin{enumerate}[label=(\alph*)]
  \item 
    If $X\supset Y$, then any $f$ vanishing on $X$ must necessarily vanish
    on all of $Y$, so $\I X\subset\I Y$.\\
    Similarly, if $I\supset J$, then any $x$ on which all $f\in J$ vanish must
    also be a point where all $f\in I$ vanish, so $\V I\subset\V J$.
  \item
    $f\in S$ means that it vanishes at all $x\in\V S$. Now, $\I{\V S}$ is the
    set of functions which vanish all over $\V S$, and all $f\in S$ do. Hence
    $\I{\V S}\supset S$.\\
    All $f\in\I X$ vanish on $X$. $\V{\I X}$ is the set of points where all $f$ vanish,
    and they certainly do on $X$.
  \item
    For all sets $X$ of points in $\Af$, $\V{\I{X'}}\supset X'$. Setting $X'=\V
    S$, we have $\I{\V{\I X}}\supset \I X$.
    We know that $X \subset\V{\I X}$. Applying $\mathcal I$ reverses
    inclusions, so that $\I{\V{\I X}}\subset\I X$. Together, we get that
    $\I{\V{\I X}} = \I X$.\\
    The argument for the other side is almost the exact dual of this, so we omit
    it. 
\end{enumerate}

\begin{block*}$\I X$ is radical for any $X\subset\Af$ .\end{block*}
\noindent We recall the definition of the radical of an ideal:
\[\sqrt I = \{a\in R : a^n\in I \text{ for some n } > 0\}\]
A radical ideal is any ideal $I$ for which $I=\sqrt I$.\\\\
The inclusion $I\subset \sqrt I$ is trivial (just take $n=1$ in the definition).
We prove the reverse.
For an algebraic set $V$, $\I V$ is the set of all polynomials which vanish
everywhere on $V$, that is:
\[\I V:=\{f\in\kn : \forall x\in V. f(x) = 0\}\]
Then
\[\sqrt{\I V}=\{f\in\kn : \forall x\in V. f(x)^n = 0\text{ for some }n>0\}\]
But if this holds for any $n$, then it holds for all $n$: in particular, it
holds for $n=1$, so this definition reduces to the previous one.

\begin{block*}Consider an algebraic set $W\subset\Af$, and $p\in\Af$ not in $W$.
  Construct a function that is $0$ on $W$ and $1$ on $p$.\end{block*}
$\I W\neq\I{W\cup \{p\}}$ (if not, then $p$ is in $W$). Hence, there is at least
one polynomial, say $g$, in the latter ideal that does not belong to the former.
The required polynomial function is $\frac{g(x)}{g(p)}$.
\section{The Hilbert basis theorem}
Apparently all algebraic sets are an intersection of a finite number of
hypersurfaces. In other words, you can define an algebraic set with a finite
number of polynomials in every case!
\begin{thm}[The Hilbert basis theorem] If $R$ is Noetherian, then $\Rn$ is
  Noetherian too. 
\end{thm}
\begin{proof}[Proof]
  Note that $\Rn$ is canonically iso to $R[x_1,x_2,\ldots,x_{n-1}][x_n]$, so if
  we show that $R$ Noetherian implies $R[x]$ Noetherian, then we are done by
  induction (the base case is obviously $R$, which is Noetherian by assumption).
  Let $I$ be an ideal of $R[x]$. We have to find a finite set of generators for
  $I$.\\
  Let $J$ be the set of leading coefficients of polynomials in $R[x]$. 
\begin{lem}$J$ is an ideal of $R$.\\
Closure under addition: if $j$ and $j'$ are two elements of $J$ corresponding to
the polynomials $f$ and $f'$ of $R[x]$, then the leading coefficient of the
polynomial $f+f'\in R[x]$ is $j+j'\in J$.\\
Closure under $R$-multiplication: If $j\in J$ corresponds to $f\in R[x]$, then for all $r\in R$, $rj$ is
the leading coefficient of $rf\in R[x]$. 
\end{lem}
Since $R$ is Noetherian, this implies that $J$ is finitely generated. Let the
polynomials in $R[x]$ whose leading coefficients generate $J$ be
$F=\{F_1,F_2,\ldots,F_r\}\subset I$, and let $N$ be an integer greater than
$\max\deg F$.\par
For each $m\leq N$, let $J_m$ be the ideal of $R$ consisting of the leading
coefficients of all $F\in R[x]$ with $\deg F\leq m$. (The argument that this is
an ideal is almost identical to that in the lemma.) Let $F_m:={F_{m_j}}_1^k$ be
a finite generating set for $J_m$. (Again, by Noetherian-ness, this exists.)\par
Crux move: Let $I'$ be the ideal generated by $F$ and all the $F_m$s. We claim
$I\subseteq I'$. Since we have, in that case, constructed a finite generating set for any
arbitrary ideal $I$, will have shown that $R[x]$ is Noetherian.\par
Assume the contrary, i.e. that there are $G\in I$ not in
$I'$. Choose the one with the least degree. There are then two cases:
\begin{enumerate}[label=(\alph*)]
  \item $\deg G > N$. We can then find some polynomials $Q_i$ such that $T =
    \sum Q_iF_i$ has the same leading coefficient as $G$ does. Now, $T$ is an
    element of $I'$, so $G-T\in I'$. Since $I'$ is an ideal, $G$ must also be in
    $I'$.
  \item $\deg G = m\leq N$, then we can lower the degree by subtracting off some
    $\sum Q_jF_{m_j}$, and $G\in I'$ again.
\end{enumerate}
In both cases, we arrive at a contradiction. Thus, no such $G$ exists, and hence
we see that $I'\supseteq I$. Since $I'$ has been shown to be finitely generated,
$I$ is too. 
\end{proof}

\begin{cor}For all fields $k$, $\kn$ is Noetherian.\end{cor}
\begin{thmsmall}Any algebraic set $V$ is the intersection of finitely many
  hypersurfaces.\end{thmsmall}

\begin{proof}
  Let $V = \V I$ for some ideal $I$ of $\kn$. As an ideal of a polynomial ring over a field, it is finitely
  generated by the previous theorem. Consider a generating set
  $F=\{F_1,F_2,\ldots,F_n\}$ for $I$. Then $V = \V I = \V {F_1}\cap\V
    {F_2}\cap\cdots\cap\V{F_n}$. Each $\V{F_i}$ is the zero set of a single
  polynomial in $\kn$, which is by definition a hypersurface.
\end{proof}

\begin{block*}Let $\pi:R\to R/I$ be the projection map from a ring onto a
  quotient ring. Show that
  \begin{enumerate}[label=(\alph*)]
    \item For every ideal $J$ containing $I$, $\pi(J)$ is an ideal of $R/I$, and
      for every ideal $J$ of $R/I$, $\pi^{-1}(J)$ is an ideal of $R$ containing
      $J$.
    \item If $J$ is mapped to $J'$, then $J$ is radical (resp. prime, maximal)
      iff $J'$ is.
    \item If $J$ is finitely generated, $\pi(J)$ is as well. Conclude that $R/I$ is
      Noetherian if $R$ is.
  \end{enumerate}
\end{block*}
  \hfill
  \begin{enumerate}[label=(\alph*)]
    \item 
      \begin{proof}
        Apply the Correspondence Theorem to $\pi$.
      \end{proof}
    \item 
      \begin{proof}[Maximal ideals]\mbox{}\\* 
        This follows trivially from the lattice
        isomorphism theorem, in both directions.
      \end{proof}
      \begin{proof}[Prime ideals]\mbox{}\\*    
        $\fr j$ prime $\implies\pi(\fr j)$ prime:\par 
        Let $\fr j$ be a prime ideal in $R$, and $a'b' =
        \bar{a}\bar{b} \in \pi(\fr j)$. Then there are $i_a,i_b$ in $I$ 
        such that $a+i_a,b+i_b\in R$.\\
        Multiply $a+i_a$ and $b+i_b$:
        \[(a+i_a)(b+i_b) = ab + (bi_a+ai_b+i_ai_b)\]
        Suppose $ab\in\fr j$. Then this expression is in $\fr j$, since 
        $\fr j\supseteq I$. Since $\fr j$ is prime, wlog $a\in\fr j$, so 
        $a+i_a$ in $\pi(\fr j)$. Hence, $\fr j$ prime implies $\pi(\fr j)$ 
        prime.\par
        $\pi(\fr j)$ prime $\implies \fr j$ prime:\par
        Let $\bar a\bar b\in\pi(\fr j)$. Then $\pi^{-1}(\bar a\bar b) 
        = ab\in\fr j$. Since $\fr j$ is prime, wlog $\bar a\in\pi(\fr j)$. 
        Then $a=\pi^{-1}(\bar a) \in\fr j$. Hence, $\fr j$ is prime.
      \end{proof}
      \begin{proof}[Radical ideals]\mbox{}\\*
        $\fr j$ radical $\implies\pi(\fr j)$ radical:\par 
        If $\fr j$ is radical, this means that $\sqrt{\fr j}\subseteq \fr
        j$. (The other direction holds trivially). Since $I\subseteq\fr j$,
        we can apply $\pi$ and use the lattice iso theorem:
        \[\pi(\sqrt{\fr j})=\sqrt{\pi(\fr j)}\subseteq\pi(\fr j),\]
        hence $\pi(\fr j)$ is a radical ideal too.\par
        $\pi(\fr j)$ radical $\implies\fr j$ radical:\par
        The proof of this is just the previous one in reverse, so we omit it.
      \end{proof}
    \item 
      \begin{proof}
        Let $J$ be finitely generated by $f_1,f_2,\ldots,f_n$. Any
        $j\in\pi(J)$ is the image (residue?) of some $k\in J$. Since there exist
        $a_i$ such that
        \[k=\sum a_if_i\]
        we can apply $\pi$ to both sides of this to get
        \[j=\pi(k)=\pi(\sum a_if_i)=\sum\pi(a_i)\pi(f_i)\]
        so the $\pi(f_i)$ form a finite generating set for $\pi(J)$. Since every
        ideal in $R/I$ is the homomorphic image of some ideal of $R$, all ideals
        of the quotient ring are finitely generated.\\
      \end{proof}
      \noindent
      \begin{proof}[Alternative proof]
        We can directly show that for any
        $\eta:R\to S$, if $R$ is Noetherian, $S':=\im R$ is as well. One way of doing this
        is the proof given above (note that it does not use the fact that we are
        mapping into the quotient ring anywhere!), but there is also another way,
        using the alternative ``ascending chain condition'' definition of
        Noetherian-ness.\par
        Let ${\mathfrak i}_1\subseteq {\mathfrak i}_2\subseteq \cdots$ be a chain of ideals in $S'$. We
        will show that this must stabilize, by using the fact that homomorphisms
        preserve inclusions (i.e. basically the lattice isomorphism theorem).\par 
        To that end, apply $\eta^{-1}$ to this chain. This gives us a chain of
        ideals in $R$ which all contain $\ker\eta$, since $\im R\cong R/\ker
        \eta$ and hence $\eta(\ker\eta) = (0)$ in $S'$. 
        This chain must stabilize since $R$ is Noetherian:
        \[\eta^{-1}({\mathfrak i}_1)\subseteq\eta^{-1}({\mathfrak i}_2)
          \subseteq\cdots \subseteq\eta^{-1}({\mathfrak i}_n)=
          \eta^{-1}({\mathfrak i}_{n+1})=\cdots\]
        This chain stabilizes at the $n$th position. Apply $\eta$ to this again,
        to get
        \[{\mathfrak i}_1\subseteq {\mathfrak i}_2\subseteq
          \cdots\subseteq {\mathfrak i}_n={\mathfrak i}_{n+1}=\cdots\]
        so this chain stabilizes as well. Hence, $\im R$ is also Noetherian.
      \end{proof}
  \end{enumerate}

\section{Irreducible components of an alg. set}
\begin{thmsmall}
  An algebraic set $V$ is irreducible iff $\I V$ is
  prime.
\end{thmsmall}
\begin{proof}
  Suppose $\I V$ is not prime. Then there is some $fg\in\I V$ (vanishing all
  over $V$) for which $f\notin\I V,g\notin\I V$ (so the factors do not vanish
  all over $\I V$). Then
  \begin{align*}
    V &= V\cap \V{\I{\V{fg}}}\\ 
      &= V\cap \V{fg}\\
      &= V\cap(\V f\cup\V g)\\
      &= (V\cap\V f)\cup(V\cap\V g)
  \end{align*}
  and neither of the last two sets are equal to $V$.\par
  In the other direction, let $V=V_1\cup V_2$. Pick some $f$ vanishing on $V_1$
  but not all over $V_2$, and a $g$ vanishing on $V_2$ but not all over $V_1$.
  Then $fg$ vanishes all over $V$, so $fg\in\I V$, but neither $f$ nor $g$ are
  in $\I V$. Hence $\I V$ is not prime.
\end{proof}

\begin{thmsmall}
  
\end{thmsmall}

\section{Algebraic subsets of the plane}
\begin{thm}[``Weak B\'ezout's theorem'']
  Let $f$ and $g$ be coprime polynomials in $k[x][y]$. Then $f$ and $g$ intersect
  in finitely many points.
\end{thm}
\begin{proof}
  Since $f$ and $g$ have no common factors in $k[x][y]$, they also have no
  common factors in $F=k(x)[y]$. Since $F$ is a PID and $f$ and $g$ are coprime,
  $(f,g)=(1)$.\par
  From this, we can say that there exist $r,s\in F$ such that $rf+sg=1$ in $F$.
  Then we can ``clear fractions'' by multiplying by some $d\in k[x][y]$, to get
  $af+bg=d$.\par
  If the point $(p,q)$ is in $\V{f,g}$, then 
  \[a(p,q)f(p,q)+b(p,q)g(p,q)=d(p).\]
  Now the LHS is zero at common zeros of $f$ and $g$. If $(p,q)$ is a zero, then
  $d(p)=0$, and $d$ has finitely many zeros. Hence the set of x-coordinates is
  finite in size. A similar argument holds for the y-coordinates (using
  $k[x](y)$ instead), so the set of common zeros is finite as well.
\end{proof}

\begin{thm}[Classification of affine varieties]
  Let $k$ be infinite. Then the irreducible affine subsets of $\Akn 2$ are:
  \begin{enumerate}[label=(\alph*)]
    \item $\Akn 2$
    \item $\emptyset$
    \item points
    \item {\bfseries{irreducible plane curves}} $\V f$, where $f$ is
  irreducible and $\V f$ is infinite.
  \end{enumerate}
\end{thm}

\begin{proof}
  Let $V$ be an irreducible affine algebraic set in $\Akn 2$. Then:
  \begin{enumerate}
    \item {\textit {V is finite.}} Either $V$ is empty, or $V$ consists of a
      countable collection of points $(x_i,y_i)$. For $\I V$ to be irreducible,
      there must only be one such point.
    \item {\textit{$\I V = (0)$.}} Then $V=\Akn 2$.
    \item Else, $\I V$ contains some nonconstant $f$. Since $\I V$ is a prime
      ideal (since $V$ is irreducible), some irreducible factor of $f$ is also in the ideal.
      Hence we may assume that $f$ is irreducible as well.\par
      We claim that $\I V = (f)$. Indeed, notice that if $G\in\I V$,
      $g\notin(f)$, then $f$ and $g$ are coprime. ($f=gh$ is ruled out because
      $f$ is irreducible, and $g=fh$ would imply that $g\in(f)$.) Then
      $V\subset\V(f,g)$ is finite, and we go back to case (a).
  \end{enumerate}
\end{proof}

\section{The Nullstellensatz}
\begin{thm}[Weak Nullstellensatz]
  If $I$ is a proper ideal in $\kn$, then $\V I\neq\emptyset$.
\end{thm}
\begin{proof}
  Assume $I$ is maximal. (This is okay because there is some maximal ideal $J$
  containing $I$ anyway, and $V(I)\supset V(J)$, so it works for this case.)
  Then $L:=\kn/I$ is a field, and $k$ can be considered a subfield of $L$.\par
  Suppose that $k=L$. Then, for each $1\leq i\leq n$, choose some $a_i\in k$
  such that $\bar{x_i}=a_i$, i.e. $x_i-a_i\in I$. Then these generate
  $(x_1-a_1,\ldots,x_n-a_n)$, which is maximal, hence equal to $I$. Hence $\V I
  = {(a_1,\ldots,a_n)}\neq\emptyset$.
\end{proof}
Of course, this hinges on the $k$ being equal to $L$. In fact, it always is:
\begin{thmsmall}
  Suppose $k$ is an algebraically closed subfield of a field $L$, and there is a
  ring homomorphism from $\kn$ onto $L$. Then $k=L$.
\end{thmsmall}
We prove this later. For now, we develop some results that we will need to do so.
\begin{thm}[``The'' Nullstellensatz]
  Let $k$ be algebraically closed, and $I$ be an ideal in $\kn$. Then $\I{\V I}
  = \sqrt I$.
\end{thm}
This basically says this: if $g$ and $f_i$ are in $\kn$, and $g$ vanishes
whenever the $f_i$ do, then there is some relation of the form
\[G^N = a_1f_1+a_2f_2+\cdots+a_nf_n,\]
for $a_i\in\kn$ and $N>0$.
\begin{proof}
  We know that $I\subset \I{\V I}$, and since $I$ is a radical ideal, that gives
  us $\sqrt I\subset\I{\V I}$.\par
  To prove the other inclusion, we use the acclaimed ``Rabinowitsch trick''. 
  Suppose that $g$ is in the ideal $\I{\V{f_1,\ldots f_r}}, f_i\in\kn$. Out of
  nowhere, we decide to set
  \[J:=(f_1,f_2,\ldots,f_n,x_{n+1}g-1)\subset k[x_1,\ldots,x_n,x_{n+1}].\]
  Now notice that $\V J\subset{\mathbb{A}_k^{n+1}}$ is empty. By the Weak Nullstellensatz,
  we see that this is not a proper ideal, and hence $(1)$. This means that $1\in
  J$, which in turn means that there is an equation
  \[a(x_{n+1}g-1)+\sum b_if_i = 1.\]
  where $a$ and $b$ are polynomials in $x_1,\ldots,x_n,x_{n+1}$.
  Let $y:=1/x_{n+1}$. Multiply both sides by a high enough power of $y$ to get
  \[y^n = a'(g-y)+\sum {b'}_if_i.\]
  where $a'$ and $b'$ are polynomials in $x_1,\ldots,x_n,y$. 
  Crux move: Substitute $y=g$.
\end{proof}

We now have a raft of definitions for our algebra-geometry
dictionary:
\begin{proof}[Corollary]
  Let $I$ be a radical ideal. Then $\I{\V I} = \sqrt I = I$. Hence radical
  ideals correspond to algebraic sets.
\end{proof}
\begin{proof}[Corollary]
  Let $I$ be a prime ideal. Then $\V I$ is irreducible. Hence prime ideals
  correspond to irreducible algebraic sets.
\end{proof}
\begin{proof}[Corollary]
  Let $f=f_1^{i_1}f_2^{i_2}\cdots f_n^{i_n}$ be a polynomial in $\kn$. Then
  $\I{\V f} = (f_1f_2\cdots f_r)$, and $\V f =
  \V{f_1}\cup\V{f_2}\cup\cdots\cup\V{f_n}$ are the irreducible components of
  $f$. Hence, irreducible polynomials correspond to irreducible hypersurfaces in
  $\Af$.
\end{proof}

\subsection{Exercises}
\begin{block*}
  Let $V$ be the set of all points in $\mathbb{A}^3_{\mathbb C}$ parametrized by
  $(t,t^2,t^3)$ where $t\in\mathbb C$. Find $\I V$ and show that $V$ is irreducible.
\end{block*}
\begin{proof}
  $(z-xy)\supset \I V$ is obvious. To prove the reverse inclusion, consider any
  point parametrized as given. Since $z=xy$ and $y=x^2$, we get
  $z=x^3,y=x^2,x=x$, so all polynomials in $\I V$ vanish on this point.\par
  Now, ${\mathbb C}[x,y,z]/(z-xy) = {\mathbb C}[x,y,xy] = {\mathbb C}[x,y]$,
  which is a domain: hence $(z-xy)$ is prime, implying that $V$ is irreducible. 
  (That might seem hand-wavy: $\overline{f(x,y,z)}\mapsto f(x,y,xy)$ gives an
  explicit isomorphism.)
\end{proof}

\begin{block*}
  Let $I:=(x^2+y^2,x^2-y^2)\subset\C[x,y]$. Find $\V I$ and $\dim_{\C}(\C[x,y]/I)$.
\end{block*}
\begin{proof}[Solution]
  If $(x,y)\in\V I=\V{x^2+y^2}\cap\V{x^2-y^2}$, then $x^2+y^2=x^2-y^2=0$, then
  $y^2=0\implies y=0$ and, further, $x=0$ as well. Hence $\V I = \{(0,0)\}$.\par 
  Direct computation shows that $\C[x,y]/I$ is the set of all linear
  combinations of $x,y$ and $xy$: hence it can be represented as
  $\C[x,y]/(x^2,y^2)$. 
\end{proof}

\section{Modules and finiteness}

\section{Integral elements}

\section{Field extensions}
\end{document}
